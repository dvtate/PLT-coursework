\documentclass{article}
\usepackage[left=1in,right=1in,top=1in,bottom=1in]{geometry}
\usepackage{amsmath, amsthm, verbatim, enumerate, xcolor, mathpartir}

\newtheorem{task}{Task}
\newtheorem{lemma}{Lemma}
\newtheorem{thm}{Theorem}
  

%% Inference rules
%Inference rule with a title
\newcommand{\Rule}[4][]{\ensuremath{\inferrule*[right={(#2)},#1]{#3}{#4}}}
%Inference rule with no title
\newcommand{\RuleNoLabel}[3][]{\ensuremath{\inferrule*[#1]{#2}{#3}}}

%% BNF
\newcommand{\bnfdef}{\mathop{::=}}
\newcommand{\bnfalt}{\mbox{\Large $\,\mid\,$}}

%% Language names
\newcommand{\Elang}{\ensuremath{\mathsf{E}}}

%% RB Trees
\newcommand{\tree}{\ensuremath{T}}
\newcommand{\leaf}{\ensuremath{\mathsf{Leaf}}}
\newcommand{\bnode}[2]{\ensuremath{\mathsf{BNode}(#1, #2)}}
\newcommand{\rnode}[2]{\ensuremath{{\color{red}{\mathsf{RNode}}}(#1, #2)}}
\newcommand{\bc}{\ensuremath{\mathsf{Black}}}
\newcommand{\rc}{\ensuremath{{\color{red}{\mathsf{Red}}}}}
\newcommand{\istree}[2]{~\ensuremath{\mathsf{isTree}~#1~#2}}
\newcommand{\colr}{\ensuremath{c}}
\newcommand{\nodes}[1]{\ensuremath{\mathit{Nodes}(#1)}}

%% E syntax
\newcommand{\kw}[1]{\ensuremath{\mathsf{#1}}}
\newcommand{\cat}{\ensuremath{\mathbin{\text{\textasciicircum}}}}
\newcommand{\kwlen}[1]{\ensuremath{|#1|}}
\newcommand{\kwlet}[3]{\ensuremath{\kw{let}~#1=#2~\kw{in}~#3}}
\newcommand{\kwn}[1]{\ensuremath{\overline{#1}}}
\newcommand{\kws}[1]{\text{``}\ensuremath{#1}\text{''}}
\newcommand{\kwif}[3]{\ensuremath{\kw{if}~#1~\kw{then}~#2~\kw{else}~#3}}
\newcommand{\kwtrue}{\kw{true}}
\newcommand{\kwfalse}{\kw{false}}

\newcommand{\kwint}{\kw{int}}
\newcommand{\kwstring}{\kw{string}}
\newcommand{\kwbool}{\kw{bool}}

%% Judgments
\newcommand{\ctx}{\ensuremath{\Gamma}}
\newcommand{\val}{\ensuremath{~\mathsf{val}}}
\newcommand{\step}{\ensuremath{\mapsto}}
\newcommand{\typed}[3]{\ensuremath{#1 \vdash #2 : #3}}
\newcommand{\sub}[3]{\ensuremath{[#1/#2]#3}}
\newcommand{\fv}[1]{\ensuremath{\mathsf{FV}(#1)}}


\title{IIT CS595: Topics \& Applications in Programming Languages\\
  {\large Homework 2: Memory and Cost Semantics}}
\author{Dustin Van Tate Testa}
\date{Out: Tuesday, Oct. 19\\
  Due: Wednesday, Nov. 3 11:59pm CDT\\
  %\vspace{0.5cm}
  %\updated{Sep. 15}
}

\begin{document}
\maketitle

%\textbf{This assignment contains~{\numwtasks} written tasks and~{\numptasks}
%  programming tasks, for a total of~{\totpoints} points.}


\section*{Logistics}
The submission and collaboration instructions (as well as the preference
for submissions typeset in LaTeX) are the same as for HW0 and HW1.

\textbf{In particular:}
Submit your answers as a single .pdf or .doc file  on Blackboard
under
the correct assignment. 

\section{Allocation and Memory Use}

As we discussed in class, cost semantics can be useful for more than just
counting the number of steps a program will take to execute.
%
In this assignment, we'll use them to track memory usage.
%
Consider this new variant of STLC:

\[
\begin{array}{r l l}
  \tau & \bnfdef & \kwunitt \bnfalt
  \kwarr{\tau}{\tau} \bnfalt
  \kwprod{\tau}{\tau}\\
  e & \bnfdef & x \bnfalt
  \kwunit \bnfalt
  \kwfun{x}{\tau}{e} \bnfalt
  \kwapp{e}{e} \bnfalt
  \kwpair{e}{e} \bnfalt
  \kwvpair{e}{e} \bnfalt
  \kwfst{e} \bnfalt
  \kwsnd{e}\\
\end{array}
\]

We now have two forms for pairs: $\kwpair{e_1}{e_2}$ and $\kwvpair{e_1}{e_2}$.
The latter form, $\kwvpair{e_1}{e_2}$, is used for a pair that has fully
evaluated and been allocated in memory (this models how structures like pairs
are executed in real languages; when they are constructed, memory for the
structure is allocated, e.g., using \texttt{malloc}).
%
This is made clearer in the dynamic semantics.
%
Like when we covered imperative languages, the state has a store~$\store$.
%
These stores are actually just lists of all of the pairs that have been
produced/allocated.
%
We'll use~$\stjoin{\store_1}{\store_2}$ to combine stores
and~$|\store|$ to refer to the size of a store (i.e., the length of the list).

The key different rules are (V-3) and (S-6).
%
Note from the (V-\#) rules that~$\kwpair{e_1}{e_2}$ is {\em not} a value,
even if~$e_1$ and~$e_2$ are values.
%
Instead, (S-6) says that when~$v_1$ and~$v_2$ are values,~$\kwpair{v_1}{v_2}$
takes an additional step to allocate the memory for the pair by adding it
to the store and steps to~$\kwvpair{v_1}{v_2}$, which is now a value%
\footnote{If we really wanted to faithfully model how this works in real
  languages, we would return a representation of the location in memory where
  the pair is stored (basically a pointer), and look it up in memory if we
  need to get the components of the pair later. The way we're doing it here
  is a little contrived but avoids a bunch of cumbersome notation I didn't
  want to burden you with.}.
%
For the same reason,~$\kwfst{e}$ and~$\kwsnd{e}$ evaluate~$e$ until
it looks like~$\kwvpair{v_1}{v_2}$: the pair must be allocated before we can
take its components and if we have a pair~$\kwvpair{v_1}{v_2}$, we know it's
been allocated.

Make sure you understand the language and the distinctions made above
before continuing.
%
If you have questions, feel free to ask for clarifications on Discord or
at office hours.

{
  \centering
  \def \MathparLineskip {\lineskip=0.43cm}
  \begin{mathpar}
    \Rule{V-1}
         {\strut}
         {\kwunit\val}
    \and
    \Rule{V-2}
         {\strut}
         {\kwfun{x}{\tau}{e}\val}
    \and
    \Rule{V-3}
         {v_1 \val\\
           v_2 \val}
         {\kwvpair{v_1}{v_2} \val}
    \and
    \Rule{S-1}
         {\cfg{\store}{e_1} \step \cfg{\store'}{e_1'}}
         {\cfg{\store}{\kwapp{e_1}{e_2}}
           \step
           \cfg{\store'}{\kwapp{e_1'}{e_2}}}
    \and
    \Rule{S-2}
         {\cfg{\store}{e_2} \step \cfg{\store'}{e_2'}}
         {\cfg{\store}{\kwapp{(\kwfun{x}{\tau}{e})}{e_2}}
           \step
           \cfg{\store'}{\kwapp{(\kwfun{x}{\tau}{e})}{e_2'}}}
    \and
    \Rule{S-3}
         {v\val}
         {\cfg{\store}{\kwapp{(\kwfun{x}{\tau}{e})}{v}}
           \step
           \cfg{\store}{\sub{v}{x}{e}}}
    \and
    \Rule{S-4}
         {\cfg{\store}{e_1} \step \cfg{\store'}{e_1'}}
         {\cfg{\store}{\kwpair{e_1}{e_2}}
           \step
           \cfg{\store'}{\kwpair{e_1'}{e_2}}}
    \and
    \Rule{S-5}
         {v_1\val\\
           \cfg{\store}{e_2} \step \cfg{\store'}{e_2'}}
         {\cfg{\store}{\kwpair{v_1}{e_2}}
           \step
           \cfg{\store'}{\kwpair{v_1}{e_2'}}}
    \and
    \Rule{S-6}
         {v_1\val\\
           v_2\val}
         {\cfg{\store}{\kwpair{v_1}{v_2}}
           \step
           \cfg{\stjoin{\store}{\stsing{\kwvpair{v_1}{v_2}}}}
               {\kwvpair{v_1}{v_2}}
         }
    \and
    \Rule{S-7}
         {\cfg{\store}{e} \step \cfg{\store'}{e'}}
         {\cfg{\store}{\kwfst{e}} \step \cfg{\store'}{\kwfst{e'}}}
    \and
    \Rule{S-8}
         {\strut}
         {\cfg{\store}{\kwfst{\kwvpair{v_1}{v_2}}}
           \step \cfg{\store}{v_1}}
    \and
    \Rule{S-9}
         {\cfg{\store}{e} \step \cfg{\store'}{e'}}
         {\cfg{\store}{\kwsnd{e}} \step \cfg{\store'}{\kwsnd{e'}}}
    \and
    \Rule{S-10}
         {\strut}
         {\cfg{\store}{\kwsnd{\kwvpair{v_1}{v_2}}}
         \step \cfg{\store}{v_2}}
\end{mathpar}
}

The cost semantics to track memory usage then should correspond not to the
number of steps a program will take to execute, but the number of pairs it
allocates, that is, the length of~$\store$ when it is done.
%
I've filed out the big-step semantics for the language for you.
%
Note that the big-step semantics {\em don't} carry around a $\store$; that's
too low-level a detail for the abstract big-step semantics.

{
  \centering
  \def \MathparLineskip {\lineskip=0.43cm}
  \begin{mathpar}
    \Rule{E-1}
         {v\val}
         {\eval[0]{v}{v}}
    \and
    \Rule{E-2}
         {\eval[n_1]{e_1}{\kwfun{x}{\tau}{e}}\\
           \eval[n_2]{e_2}{v}\\
           \eval[n_3]{\sub{v}{x}{e}}{v'}}
         {\eval[n_1 + n_2 + n_3]{\kwapp{e_1}{e_2}}{v'}}
    \and
    \Rule{E-3}
         {\eval[n_1]{e_1}{v_1}\\
           \eval[n_2]{e_2}{v_2}
         }
         {\eval[n_1 + n_2 + 1]{\kwpair{e_1}{e_2}}{\kwvpair{v_1}{v_2}}}
    \and
    \Rule{E-4}
         {\eval[n]{e}{\kwvpair{v_1}{v_2}}}
         {\eval[n]{\kwfst{e}}{v_1}}
    \and
    \Rule{E-5}
         {\eval[n]{e}{\kwvpair{v_1}{v_2}}}
         {\eval[n]{\kwsnd{e}}{v_2}}
  \end{mathpar}
}

\begin{task}
  Fill in the blanks in the big-step semantics to turn it into a cost semantics
  for tracking memory usage.
  %
  If you're having trouble deciding exactly what operations to ``charge'' for,
  look ahead at the theorems we want to prove about the cost semantics.
  %
  (Conversely, if you're having trouble proving the results later, you
  might want to come back and re-evaluate whether you need to change the
  cost semantics.)
\end{task}

We'll now prove that the cost semantics corresponds to the memory used by
the small-step semantics.
%
That is:
\[
\eval[n]{e}{v} \Leftrightarrow (\cfg{\estore}{e} \step^* \cfg{\store}{v}
\text{~and~} n = |\store|)
\]
where~$|\store|$ is the size of the store and~$\estore$ is the empty store.
%
As you'd expect, $|\estore| = 0$.

We prove this in several pieces.
%
Recall that the forward direction involves proving several annoying lemmas.
%
Here's the first annoying lemma.
%
It says that if~$e \step^* e'$ and $e$ is actually a part of a bigger
expression, that bigger expression also steps appropriately.
%
It has 6 parts for each kind of ``bigger expression.''

\begin{lemma}\label{lem:compose-exps}
  \begin{enumerate}
  \item
    If~$\cfg{\store}{e_1} \step^* \cfg{\store'}{e_1'}$
    then~$\cfg{\store}{\kwapp{e_1}{e_2}} \step^*
    \cfg{\store'}{\kwapp{e_1'}{e_2}}$.
  \item
    If~$\cfg{\store}{e_2} \step^* \cfg{\store'}{e_2'}$
    then~$\cfg{\store}{\kwapp{e_1}{e_2}} \step^*
    \cfg{\store'}{\kwapp{e_1}{e_2'}}$.
  \item
    If~$\cfg{\store}{e_1} \step^* \cfg{\store'}{e_1'}$
    then~$\cfg{\store}{\kwpair{e_1}{e_2}} \step^*
    \cfg{\store'}{\kwpair{e_1'}{e_2}}$.
  \item
    If~$\cfg{\store}{e_2} \step^* \cfg{\store'}{e_2'}$
    then~$\cfg{\store}{\kwpair{e_1}{e_2}} \step^*
    \cfg{\store'}{\kwpair{e_1}{e_2'}}$.
  \item
    If~$\cfg{\store}{e} \step^* \cfg{\store'}{e'}$
    then~$\cfg{\store}{\kwfst{e}} \step^*
    \cfg{\store'}{\kwfst{e}}$.
  \item
    If~$\cfg{\store}{e} \step^* \cfg{\store'}{e'}$
    then~$\cfg{\store}{\kwsnd{e}} \step^*
    \cfg{\store'}{\kwsnd{e}}$.
  \end{enumerate}
\end{lemma}


To do the proof, recall that the~$\step^*$ judgment is also defined
inductively:

{
  \centering
  \begin{mathpar}
    \Rule{M-1}
         {\strut}
         {\cfg{\store}{e} \step^* \cfg{\store}{e}}
    \and
    \Rule{M-2}
         {\cfg{\store}{e} \step \cfg{\store''}{e''}\\
           \cfg{\store''}{e''} \step^* \cfg{\store'}{e'}}
         {\cfg{\store}{e} \step^* \cfg{\store'}{e'}}
  \end{mathpar}
}

The parts are all reasonably similar.
I'll prove one, you'll prove another.

\begin{proof}
  \begin{enumerate}
  \item
    By induction on the derivation of
    $\cfg{\store}{e_1} \step^* \cfg{\store'}{e_1'}$.
    \begin{itemize}
    \item (M-1). Then~$e_1 = e_1'$ and~$\store = \store'$.
      By (M-1),
      $\cfg{\store}{\kwapp{e_1}{e_2}} \step^*
      \cfg{\store}{\kwapp{e_1}{e_2}}$.
    \item (M-2)
      Then~$\cfg{\store}{e_1} \step \cfg{\store''}{e_1''}$
      and $\cfg{\store''}{e_1''} \step^* \cfg{\store'}{e_1'}$.
      By (S-1),
      $\cfg{\store}{\kwapp{e_1}{e_2}} \step \cfg{\store''}{\kwapp{e_1''}{e_2}}$.
      By induction,
      $\cfg{\store''}{\kwapp{e_1''}{e_2}} \step^* \cfg{\store'}{\kwapp{e_1'}{e_2}}$
      so by (M-2),
      $\cfg{\store}{\kwapp{e_1}{e_2}} \step^* \cfg{\store'}{\kwapp{e_1'}{e_2}}$.
    \end{itemize}
  \end{enumerate}
\end{proof}

\begin{task}
  Prove part 5 of Lemma~\ref{lem:compose-exps}.
\end{task}

\begin{proof}
  By induction on the derivation of
    $\cfg{\store}{e} \step^* \cfg{\store'}{e'}$.
    \begin{itemize}
    \item (M-1). Then~$e = e'$ and~$\store = \store'$.
      By (M-1),
      $\cfg{\store}{\kwfst{e}} \step^*
      \cfg{\store}{\kwfst{e}}$.
    \item (M-2)
      Then~$\cfg{\store}{e} \step \cfg{\store''}{e''}$ 
      and $\cfg{\store''}{e''} \step^* \cfg{\store}{e'}$.\\
      By (S-7), 
      $\cfg{\store}{\kwfst{e}} \step \cfg{\store''}{\kwfst{e''}}$.
      By induction,
      $\cfg{\store''}{\kwfst{e''}} \step^* \cfg{\store'}{\kwfst{e'}}$
      so by (M-2),
      $\cfg{\store}{\kwfst{e}} \step^* \cfg{\store'}{\kwfst{e'}}$
    \end{itemize}
\end{proof}
  
We also need to know that if we have two sequences of steps, we can ``glue''
them together.

\begin{lemma}\label{lem:compose-steps}
  If~$\cfg{\store}{e} \step^* \cfg{\store'}{e'}$
  and~$\cfg{\store'}{e'} \step^* \cfg{\store''}{e''}$
  then~$\cfg{\store}{e} \step^* \cfg{\store''}{e''}$.
\end{lemma}

\begin{task}
  Prove Lemma~\ref{lem:compose-steps}.
  You have two choices of what to induct on. Take an educated guess based on
  the rules you'll have to apply to conclude that
  $\cfg{\store}{e} \step^* \cfg{\store''}{e''}$.
  If the proof isn't working out, go back and try inducting on the other
  assumption.
\end{task}

\begin{proof}
  Your proof here.
  \begin{itemize}
    \item by induction on derivation of the step judgement, either
        \begin{itemize}
            \item $e = e' = e''$ and $\store = \store' = \store''$:\\
                in which case for both step judgements (M-1) would be applied
                
            \item $e' = e''$ and $\store' = \store''$:\\
                in which case (M-2) is applied to the first step and (M-1) applied after
                % \begin{mathpar}
                %     \inferrule*
                %     {\inferrule* {$ $}{\cfg{\store}{e} \step \cfg{\store'''}{e'''}}\\
                %       \inferrule* {$ $}{\cfg{\store'''}{e'''} \step^* \cfg{\store'}{e'}}}
                %     {\cfg{\store}{e} \step^* \cfg{\store'}{e'}}
                % \end{mathpar}
            \item $e = e'$ and $\store = \store'$:\\
                in which case (M-1) is applied to make the first step and then (M-2) after
                
            \item all $e$ and all $\store$ values are unique:\\
                in which case (M-2) is applied twice
        \end{itemize}
  \end{itemize}
\end{proof}

Now we can prove the forward direction of the theorem.
%
Note that we don't actually prove
that~$\cfg{\estore}{e} \step^* \cfg{\store}{v}$.
%
We actually need to be able to start with a non-empty store in order to
have a strong enough induction hypothesis.
%
Think about why.
%
We also prove that when~$e$ steps to~$v$ starting with store~$\store$,
the resulting store is strictly an extension of~$\store$: that is,
it's~$\stjoin{\store}{\store'}$, where~$\store'$ contains what we've added
to the store while stepping.
%
This means you don't have to worry about tracking things getting de-allocated
or changed (we don't have garbage collection.)

\begin{thm}\label{thm:forward}
  If $\eval[n]{e}{v}$ then for all~$\store$, it is the case that
  $\cfg{\store}{e} \step^* \cfg{\stjoin{\store}{\store'}}{v}$
  and $|\store'| = n$.
\end{thm}

I'll prove one case as an example.

\begin{proof}
  By induction on the derivation of $\eval[n]{e}{v}$.
  \begin{itemize}
    
  \item (E-2).
    Then~$e = \kwapp{e_1}{e_2}$
    and~$\eval[n_1]{e_1}{\kwfun{x}{\tau}{e}}$
    and~$\eval[n_2]{e_2}{v}$
    and~$\eval[n_3]{\sub{v}{x}{e}}{v'}$ and~$n = n_1 + n_2 + n_3$.
    By induction,
    \[
    \cfg{\store}{e_1} \step^*
    \cfg{\stjoin{\store}{\store_1}}{\kwfun{x}{\tau}{e}}
    \text{~and~}\cfg{\stjoin{\store}{\store_1}}{e_2} \step^*
    \cfg{\stjoin{\stjoin{\store}{\store_1}}{\store_2}}{v}
    \text{~and~}\cfg{\stjoin{\stjoin{\store}{\store_1}}{\store_2}}
    {\sub{v}{x}{e}} \step^*
    \cfg{\stjoin{\stjoin{\stjoin{\store}{\store_1}}{\store_2}}{\store_3}}
        {v'}
    \]
    and
    \[|\store_1| = n_1\text{~and~}|\store_2| = n_2\text{~and~}|\store_3| = n_3\]
    By Lemma~\ref{lem:compose-exps},
    $\cfg{\store}{e} \step^*
    \cfg{\stjoin{\store}{\store_1}}{\kwapp{(\kwfun{x}{\tau}{e})}{e_2}}$
    and~$\cfg{\stjoin{\store}{\store_1}}{\kwapp{(\kwfun{x}{\tau}{e})}{e_2}}
    \step^*
    \cfg{\stjoin{\stjoin{\store}{\store_1}}{\store_2}}
          {\kwapp{(\kwfun{x}{\tau}{e})}{v}}$.\\
          By Lemma~\ref{lem:compose-steps},
      $\cfg{\store}{e} \step^*
        \cfg{\stjoin{\stjoin{\store}{\store_1}}{\store_2}}
            {(\kwapp{\kwfun{x}{\tau}{e})}{v}}$.\\
      By (S-3) and Lemma~\ref{lem:compose-steps},
      $\cfg{\store}{e} \step^*
      \cfg{\stjoin{\stjoin{\stjoin{\store}{\store_1}}{\store_2}}{\store_3}}
          {v'}$.
          We have
      $|\stjoin{\store_1}{\stjoin{\store_2}{\store_3}}| =
      |\store_1| + |\store_2| + |\store_3| = n$.

\item (E-1). Then $e \val$ and $\eval[0]{e}{v}$ and $e = v$ and $n = 0$ which by lemma 1 translates to $\cfg{\estore}{e} \step^* \cfg{\store}{e}$. Because values cannot step according to the defined small-step semantics rules (exhaustion of rules (V- 1..3)), the $\step^*$ judgement must follow rule (M-1), thus $\estore = \store$ and $|\store| = n = 0$.

\item (E-3).
    Then~$e = \kwpair{e_1}{e_2}$
    and~$\eval[n_1]{e_1}{v_1}$
    and~$\eval[n_2]{e_2}{v_2}$
    and~$\eval[1]{\kwpair{v_1}{v_2}}{\kwvpair{v_1}{v_2}}$
    and~$n = n_1 + n_2 + 1$.
    By induction,
    \[
    \cfg{\store}{e_1} 
    \step^* \cfg{\stjoin{\store}{\store_1}}{v_1}
        \text{~and~}
    \cfg{\store}{e_2}
    \step^* \cfg{\stjoin{\stjoin{\store}{\store_1}}{\store_2}}{v_2}
        \text{~and~}
    \cfg{\store}{e_2} 
    \step^* \cfg{\stjoin{\stjoin{\stjoin{\store}{\store_1}}{\store_2}}{\store_3}}{v_2}
    \]
    and
    \[|\store_1| = n_1\text{~and~}|\store_2| = n_2\text{~and~}|\store_3| = 1\]
    By Lemma~\ref{lem:compose-exps},
    $
        \cfg{\store}{e} 
        \step^* \cfg{\stjoin{\store}{\store_1}}{\kwpair{v_1}{e_1}}
    $and~$
        \cfg{\stjoin{\store}{\store_1}}{\kwpair{v_1}{e_1}}
        \step^* \cfg{\stjoin{\stjoin{\store}{\store_1}}{\store_2}}{\kwpair{v_1}{v_2}}
    $.\\
    By Lemma~\ref{lem:compose-steps},
      $\cfg{\store}{e} 
        \step^*
         \cfg{\stjoin{\stjoin{\store}{\store_1}}{\store_2}}{\kwpair{v_1}{v_2}}$.\\
    By (S-6) and Lemma~\ref{lem:compose-steps},
      $\cfg{\store}{e} \step^*
      \cfg{\stjoin{\stjoin{\stjoin{\store}{\store_1}}{\store_2}}{\stsing{\kwvpair{v_1}{v_2}}}}
          {\kwvpair{v_1}{v_2}}$.
          We have
      $|\stjoin{\store_1}{\stjoin{\store_2}{\stsing{\kwvpair{v_1}{v_2}}}}| =
      |\store_1| + |\store_2| + |\stsing{\kwvpair{v_1}{v_2}}| = n$.

\item (E-4).
    Then~$e = \kwfst{e_1}$
    and~$\eval[n_1]{e_1}{\kwvpair{v_1}{v_2}}$
    and~$\eval[0]{e}{v_1}$
    and~$n = n_1 + 0$.
    By induction,
    \[
    \cfg{\store}{e_1} \step^* \cfg{\stjoin{\store}{\store_1}}{\kwvpair{v_1}{v_2}}
    \text{~and~}
    \cfg{\stjoin{\store}{\store_1}}{\kwfst{\kwvpair{v_1}{v_2}}} 
        \step^* 
        \cfg{\stjoin{\stjoin{\store}{\store_1}}{\store_2}}{v_1}
    \]
    and
    \[|\store_1| = |\store_2| = n_1 \text{~and~} n = n_1 \]
    By Lemma~\ref{lem:compose-exps},
    $
        \cfg{\store}{e} \step^* \cfg{\stjoin{\store}{\store_1}}{\kwfst{\kwvpair{v_1}{v_2}}}
    $~and~$
        \cfg{\stjoin{\store}{\store_1}}{\kwfst{\kwvpair{v_1}{v_2}}} 
        \step^* 
        \cfg{\stjoin{\stjoin{\store}{\store_1}}{\store_2}}{v_1}
    $.\\
    By Lemma~\ref{lem:compose-steps},
      $\cfg{\store}{e} \step^*
        \cfg{\stjoin{\stjoin{\store}{\store_1}}{\store_2}}{v_1}$.\\
    By (S-8) and Lemma~\ref{lem:compose-steps},
      $\cfg{\store}{e} \step^*
      \cfg{\stjoin{\stjoin{\store}{\store_1}}{\store_2}}{v_1}$.
          We have
      $|\stjoin{\store_1}{\store_2}| =
      |\store_1| = n$.

\item (E-5).
    Then~$e = \kwsnd{e_1}$
    and~$\eval[n_1]{e_1}{\kwvpair{v_1}{v_2}}$
    and~$\eval[0]{e}{v_1}$
    and~$n = n_1 + 0$.
    By induction,
    \[
    \cfg{\store}{e_1} \step^* \cfg{\stjoin{\store}{\store_1}}{\kwvpair{v_1}{v_2}}
    \text{~and~}
    \cfg{\stjoin{\store}{\store_1}}{\kwsnd{\kwvpair{v_1}{v_2}}} 
        \step^* 
        \cfg{\stjoin{\stjoin{\store}{\store_1}}{\store_2}}{v_2}
    \]
    and
    \[|\store_1| = |\store_2| = n_1 \text{~and~} n = n_1 \]
    By Lemma~\ref{lem:compose-exps},
    $
        \cfg{\store}{e} \step^* \cfg{\stjoin{\store}{\store_1}}{\kwsnd{\kwvpair{v_1}{v_2}}}
    $~and~$
        \cfg{\stjoin{\store}{\store_1}}{\kwsnd{\kwvpair{v_1}{v_2}}} 
        \step^* 
        \cfg{\stjoin{\stjoin{\store}{\store_1}}{\store_2}}{v_2}
    $.\\
    By Lemma~\ref{lem:compose-steps},
      $\cfg{\store}{e} \step^*
        \cfg{\stjoin{\stjoin{\store}{\store_1}}{\store_2}}{v_2}$.\\
    By (S-9) and Lemma~\ref{lem:compose-steps},
      $\cfg{\store}{e} \step^*
      \cfg{\stjoin{\stjoin{\store}{\store_1}}{\store_2}}{v_2}$.
          We have
      $|\stjoin{\store_1}{\store_2}| =
      |\store_1| = n$.

  \end{itemize}
\end{proof}

\begin{task}
  Prove the remaining cases of Theorem~\ref{thm:forward}.
  There are many similarities between the cases, but some small, important
  differences. Copying and pasting between cases is allowed and encouraged,
  but make sure you know what to change!

  \textbf{Note:} I actually proved the {\em second} case; you still have
  to go back and prove (E-1) as well as (E-3) through (E-5).
\end{task}

The reverse direction is pretty straightforward given a key lemma introduced
in class.
%
The lemma says that if~$\eval[n]{e'}{v}$ and~$e$ steps to~$e'$,
then~$\eval[n+k]{e}{v}$, where~$k$ counts any pairs allocated by that step.

\begin{lemma}\label{lem:closed}
  If~$\eval[n]{e'}{v}$
  and~$\cfg{\store}{e} \step \cfg{\stjoin{\store}{\store'}}{e'}$
  then~$\eval[n+k]{e}{v}$
  and~$|\store'| = k$.
\end{lemma}
  \begin{proof}
    By induction on the derivation of 
    $\cfg{\store}{e} \step \cfg{\stjoin{\store}{\store'}}{e'}$.
    \begin{itemize}
    \item (S-1)
      Then~$e = \kwapp{e_1}{e_2}$ and~$e' = \kwapp{e_1'}{e_2}$.
      By inversion on (E-2),
      $\eval[n_1]{e_1'}{\kwfun{x}{\tau}{e}}$
      and~$\eval[n_2]{e_2}{v}$
      and~$\eval[n_3]{\sub{v}{x}{e}}{v'}$
      and~$n = n_1 + n_2 + n_3$.
      By induction,
      $\eval[n_1 + k]{e_1}{\kwfun{x}{\tau}{e}}$
      and~$|\store'| = k$.
      By (E-2),
      $\eval[n+k]{e}{v'}$.

    \item (S-2) Then~$e = \kwapp{\kwfun{x}{\tau}{e_1}}{e_2}$
        and~$e' = \kwapp{\kwfun{x}{\tau}{e_1}}{e_2'}$.
        By inversion on (E-2),
        $\eval[0]{\kwfun{x}{\tau}{e_1}}{\kwfun{x}{\tau}{e_1}}$
        and~$\eval[n_2]{e_2}{v'}$
        and~$\eval[n_3]{\sub{v'}{x}{e_1}}{v}$
        and~$n = 0 + n_2 + n_3$.
        By induction,
        $\eval[n' + k]{e_2}{v'}$
        and~$|\store'| = k$.
        By (E-2),
        $\eval[n+k]{e}{v}$.
        
    \item (S-3) Then~$e = \kwapp{\kwfun{x}{\tau}{e_1}}{v'}$
        and~$e' = \kwapp{\kwfun{x}{\tau}{e_1}}{e_2'}$
        and~$v' \val$.
        By inversion on (E-2),
        $\eval[0]{\kwfun{x}{\tau}{e_1}}{\kwfun{x}{\tau}{e_1}}$
        and~$\eval[0]{v'}{v'}$
        and~$\eval[n_1]{\sub{v'}{x}{e_1}}{v}$
        and~$n = 0 + 0 + n_1$.
        By induction,
        $\eval[n' + k]{\sub{v'}{x}{e_1}}{v}$
        $|\store'| = k$.
        By (E-2),
        $\eval[n+k]{e}{v}$.
    
    \item (S-4) Then~$e = \kwpair{e_1}{e_2}$
        and~$e' = \kwpair{e_1'}{e_2}$.
        By inversion on (E-3),
        $\eval[n_1]{e_1}{v_1}$
        and~$\eval[n_2]{e_2}{v_2}$
        and~$\eval[1]{\kwpair{v_1}{v_2}}{\kwvpair{v_1}{v_2}}$
        and~$n = n_1 + n_2 + 1$.
        By induction,
        $\eval[n_1 + k]{e_1}{v_1}$
        $|\store'| = k$.
        By (E-3),
        $\eval[n+k]{e}{v}$
    \item (S-5) Then~$e = \kwpair{v_1}{e_2}$
        and~$e' = \kwpair{v_1}{e_2'}$.
        By inversion on (E-3),
        $\eval[0]{v_1}{v_1}$
        and~$\eval[n_2]{e_2}{v_2}$
        and~$\eval[1]{\kwpair{v_1}{v_2}}{\kwvpair{v_1}{v_2}}$
        and~$n = 0 + n_2 + 1$.
        By induction,
        $\eval[n_2 + k]{e_2}{v_2}$
        $|\store'| = k$.
        By (E-3),
        $\eval[n+k]{e}{v}$
    \item (S-6) Then~$e = \kwpair{v_1}{v_2}$
        and~$e' = \kwpair{v_1}{v_2}$.
        By inversion on (E-3),
        $\eval[0]{v_1}{v_1}$
        and~$\eval[0]{v_2}{v_2}$
        and~$\eval[1]{\kwpair{v_1}{v_2}}{\kwvpair{v_1}{v_2}}$
        and~$n = 0 + 0 + 1$.
        By induction,
        $\eval[n_2 + k]{e}{\kwvpair{v_1}{v_2}}$
        $|\store'| = k = 1$.
        By (E-3),
        $\eval[n+k]{e}{v}$
        
    \item (S-7) Then~$e = \kwfst{e_1}$ and~$e' = \kwfst{e_1'}$.
        By inversion on (E-4),
        $\eval[n_1]{e_1}{\kwvpair{v_1}{v_2}}$
        and~$\eval[n_1]{\kwfst{e_1}}{v_1}$
        and~$n = n_1$.
        By induction,
        $\eval[n_1 + k]{e}{v_1}$
        $|\store'| = k$.
        By (E-4),
        $\eval[n+k]{e}{v}$
    
    \item (S-8) Then~$e = \kwfst{\kwvpair{v_1}{v_2}}$ and~$e' = v_1$.
        By inversion on (E-4),
        $\eval[0]{\kwvpair{v_1}{v_2}}{\kwvpair{v_1}{v_2}}$
        $\eval[0]{e'}{v_1}$, so $n = 0$.
        By (S-8), $\store' = \estore$, so $k = 0$.
        By (E-4),
        $\eval[n+k]{e}{v}$
    
    \item (S-9) Then~$e = \kwsnd{e_1}$ and~$e' = \kwsnd{e_1'}$.
        By inversion on (E-4),
        $\eval[n_1]{e_1}{\kwvpair{v_1}{v_2}}$
        and~$\eval[n_1]{\kwfst{e_1}}{v_2}$
        and~$n = n_1$.
        By induction,
        $\eval[n_1 + k]{e}{v_1}$
        $|\store'| = k$.
        By (E-5),
        $\eval[n+k]{e}{v}$
    
    \item (S-10) Then~$e = \kwfst{\kwvpair{v_1}{v_2}}$ and~$e' = v_1$.
        By inversion on (E-5),
        $\eval[0]{\kwvpair{v_1}{v_2}}{\kwvpair{v_1}{v_2}}$
        $\eval[0]{e'}{v_2}$, so $n = 0$.
        By (S-10), $\store' = \estore$, so $k = 0$.
        By (E-5),
        $\eval[n+k]{e}{v}$
    
    \end{itemize}
  \end{proof}

\begin{task}
  Prove the remaining cases of Lemma~\ref{lem:closed}.
  There are many similarities between the cases, but some small, important
  differences. Copying and pasting between cases is allowed and encouraged,
  but make sure you know what to change!
\end{task}

Given that lemma, the theorem is pretty straightforward to prove.
%
I'll just do it so you can be done with the homework.

\begin{thm}
  If $\cfg{\store}{e} \step^* \cfg{\stjoin{\store}{\store'}}{v}$ and~$v \val$,
  then~$\eval[n]{e}{v}$ and~$n = |\store'|$.
\end{thm}
\begin{proof}
  By induction on the derivation of $\cfg{\estore}{e} \step^* \cfg{\store}{v}$.
  If~$e = v,$ then $\store = \estore$ and by (E-1), $\eval[0]{v}{v}$. 
  Otherwise,
  $\cfg{\estore}{e} \step \cfg{\stjoin{\estore}{\store_1}}{e'}
  \step^* \cfg{\stjoin{\estore}{\stjoin{\store_1}{\store_2}}}{v}$.
  By induction,~$\eval[n]{e'}{v}$
  and~$n = |\store_2|$.
  By Lemma~\ref{lem:closed}, $\eval[n + k]{e}{v}$
  where~$k = |\store_1|$.
  We have~$n + k = |\stjoin{\store_1}{\store_2}|$.
\end{proof}

\end{document}
