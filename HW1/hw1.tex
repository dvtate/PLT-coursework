\documentclass{article}
\usepackage[left=1in,right=1in,top=1in,bottom=1in]{geometry}
\usepackage{amsmath, amsthm, verbatim, enumerate, xcolor, mathpartir}

\newtheorem{task}{Task}
\newtheorem{lemma}{Lemma}
\newtheorem{thm}{Theorem}
  

%% Inference rules
%Inference rule with a title
\newcommand{\Rule}[4][]{\ensuremath{\inferrule*[right={(#2)},#1]{#3}{#4}}}
%Inference rule with no title
\newcommand{\RuleNoLabel}[3][]{\ensuremath{\inferrule*[#1]{#2}{#3}}}

%% BNF
\newcommand{\bnfdef}{\mathop{::=}}
\newcommand{\bnfalt}{\mbox{\Large $\,\mid\,$}}

%% Language names
\newcommand{\Elang}{\ensuremath{\mathsf{E}}}

%% RB Trees
\newcommand{\tree}{\ensuremath{T}}
\newcommand{\leaf}{\ensuremath{\mathsf{Leaf}}}
\newcommand{\bnode}[2]{\ensuremath{\mathsf{BNode}(#1, #2)}}
\newcommand{\rnode}[2]{\ensuremath{{\color{red}{\mathsf{RNode}}}(#1, #2)}}
\newcommand{\bc}{\ensuremath{\mathsf{Black}}}
\newcommand{\rc}{\ensuremath{{\color{red}{\mathsf{Red}}}}}
\newcommand{\istree}[2]{~\ensuremath{\mathsf{isTree}~#1~#2}}
\newcommand{\colr}{\ensuremath{c}}
\newcommand{\nodes}[1]{\ensuremath{\mathit{Nodes}(#1)}}

%% E syntax
\newcommand{\kw}[1]{\ensuremath{\mathsf{#1}}}
\newcommand{\cat}{\ensuremath{\mathbin{\text{\textasciicircum}}}}
\newcommand{\kwlen}[1]{\ensuremath{|#1|}}
\newcommand{\kwlet}[3]{\ensuremath{\kw{let}~#1=#2~\kw{in}~#3}}
\newcommand{\kwn}[1]{\ensuremath{\overline{#1}}}
\newcommand{\kws}[1]{\text{``}\ensuremath{#1}\text{''}}
\newcommand{\kwif}[3]{\ensuremath{\kw{if}~#1~\kw{then}~#2~\kw{else}~#3}}
\newcommand{\kwtrue}{\kw{true}}
\newcommand{\kwfalse}{\kw{false}}

\newcommand{\kwint}{\kw{int}}
\newcommand{\kwstring}{\kw{string}}
\newcommand{\kwbool}{\kw{bool}}

%% Judgments
\newcommand{\ctx}{\ensuremath{\Gamma}}
\newcommand{\val}{\ensuremath{~\mathsf{val}}}
\newcommand{\step}{\ensuremath{\mapsto}}
\newcommand{\typed}[3]{\ensuremath{#1 \vdash #2 : #3}}
\newcommand{\sub}[3]{\ensuremath{[#1/#2]#3}}
\newcommand{\fv}[1]{\ensuremath{\mathsf{FV}(#1)}}


\title{IIT CS595: Topics \& Applications in Programming Languages\\
  {\large Homework 1: Curry-Howard, Expressions continued}}
\author{Your name here}

\begin{document}
\maketitle

\section*{Logistics}
The submission and collaboration instructions (as well as the preference
for submissions typeset in LaTeX) are the same as for HW0.

\textbf{In particular:}
Submit your answers as a single .pdf or .doc file  on Blackboard
under
the correct assignment. 

\section{The Curry-Howard Correspondence}

We discussed the Curry-Howard Correspondence in class as a correspondence
between STLC and Intuitionistic Propositional Logic (IPL).
%
We can also add universal quantification to both.
%
On the logic side, this gives us intuitionistic second-order logic,
where you can express
things like $\forall A. \forall B. A \land B \limp B \land A$
(``for all propositions $A$ and $B$, $A$ and $B$ implies $B$ and $A$'').
%
On the PL side, it gives us STLC with ``forall'' types (or, equivalently,
System F with notations for sums and products defined).
%
The type corresponding to the above logic statement would be
$\kwforall{A}{\kwforall{B}}{\kwarr{(\kwprod{A}{B})}{(\kwprod{B}{A})}}$---
the logical~$\forall$ and its type equivalent even conveniently use the
same symbol.

In Classical Logic, we have the bi-implication
$\forall A. \forall B. (A \limp B) \lbimp (\lnot A \lor B)$.
%
Only one direction of this implication is true constructively (i.e., in
intuitionistic second-order logic).


\begin{task}\label{task:ch-type}
  Write the STLC (with~$\forall$) type corresponding to
  the logical implication $\forall A. \forall B. (\lnot A \lor B) \limp (A \limp B)$.
\end{task}

\textbf{Answer:}

\begin{task}
  Prove the implication above by giving an STLC term of the type you wrote in
  Task~\ref{task:ch-type}.
\end{task}

\textbf{Answer:}

It turns out that, for any classically true proposition~$P$, $\lnot \lnot P$
is true constructively (even if~$P$ is not)\footnote{If you're wondering
  whether this fact itself is meaningful in programming languages through
  the Curry-Howard correspondence, Google ``continuation passing transformation''.}.

\begin{task}
  We've discussed that the Law of the Excluded Middle (now restated with
  $\forall$),
  $\forall A. A \lor \lnot A$, is
  not true constructively. Show that
  $\lnot \lnot (\forall A. A \lor \lnot A)$ \textit{is} true constructively by
  giving an STLC term of type
  
  \[\kwforall{\alpha}{\kwarr{(\kwarr{(\kwsum{\alpha}{(\kwarr{\alpha}{\kwvoid})})}{\kwvoid})}{\kwvoid}}\]

  \textbf{Hint: }
    The term will be of the form
    \[\kwtfun{\alpha}{\kwfun{f}{(\kwarr{(\kwsum{\alpha}{(\kwarr{\alpha}{\kwvoid})})}{\kwvoid})}{~e}}\]
    for some $e$ such that~$\ftyped{\alpha}{\ctx}{e}{\kwvoid}$,
    where~$\ctx = f:(\kwarr{(\kwsum{\alpha}{(\kwarr{\alpha}{\kwvoid})})}{\kwvoid})$.
    Can you construct terms~$e_1$ and $e_2$ such that
    $\ftyped{\alpha}{\ctx}{e_1}{\kwarr{(\kwarr{\alpha}{\kwvoid})}{\kwvoid}}$
    and~$\ftyped{\alpha}{\ctx}{e_2}{\kwarr{\alpha}{\kwvoid}}$?
    How can you then use~$e_1$ and~$e_2$ to construct the function body,~$e$?
\end{task}

\textbf{Answer:}

\section{Division by zero}

In the past, we've used type systems and the ideas of type safety (progress
and preservation) to prevent run-time type errors.
%
Here, we'll use a type system to prevent another run-time error: division by
zero.
%
Consider a language we'll call~\ezlang{}, whose syntax and dynamics are below.

{
  \centering
  \[
  \begin{array}{l l l}
    \tau & \bnfdef & \kwnz \bnfalt \kwmz\\
    e & \bnfdef &
    \kwn{n} \bnfalt
    e + e \bnfalt
    e - e \bnfalt
    e * e \bnfalt
    e / e
  \end{array}
  \]
}

{
  \centering
  \def \MathparLineskip {\lineskip=0.43cm}
  \begin{mathpar}

    \Rule{V-1}
         {\strut}
         {\kwn{n} \val}
    \and
    \Rule{S-1}
         {e_1 \step e_1'\\
         \text{\textit{op}} \in \{+, -, *, /\}}
         {e_1 \mathbin{\text{\textit{op}}} e_2 \step
           e_1' \mathbin{\text{\textit{op}}} e_2}
    \and
    \Rule{S-2}
         {e_2 \step e_2'\\
         \text{\textit{op}} \in \{+, -, *, /\}}
         {\kwn{n_1} \mathbin{\text{\textit{op}}} e_2 \step
           \kwn{n_1} \mathbin{\text{\textit{op}}} e_2'}
   \and
   \Rule{S-3}
        {\strut}
        {\kwn{n_1} + \kwn{n_2} \step \kwn{n_1 + n_2}}
   \and
   \Rule{S-4}
        {\strut}
        {\kwn{n_1} - \kwn{n_2} \step \kwn{n_1 - n_2}}
   \and
   \Rule{S-5}
        {\strut}
        {\kwn{n_1} * \kwn{n_2} \step \kwn{n_1 n_2}}
   \and     
   \Rule{S-6}
        {n_2 \neq 0}
        {\kwn{n_1} / \kwn{n_2} \step \kwn{n_1 / n_2}}
  \end{mathpar}
}

This is basically our expression language {\elang} from class
(without~$\kw{let}$), but with only integers and with subtraction,
multiplication and (integer) division added.
%
To save space and time, we have only two search rules: rules (S-1) and
(S-2) apply for all~4 arithmetic operations in place of \textit{op}.
%
Note that rule (S-6), as we'd hope, allows division only if~$\kwn{n_2}$ is
nonzero.
%
That means that a division by zero, e.g.,~$\kwn{1} / \kwn{0}$, is a ``stuck''
expression: it's not a value and it can't step.
%
So in order to make Progress and Preservation hold, we need a type system
that rules out division by zero.
%
The {\ezlang} language has only integers, so we don't need types to
distinguish integers from strings, but we still have two
types:~$\kwnz$ (``not zero'') and~$\kwmz$ (``maybe zero''), which we'll use to
check whether an expression {\em might} evaluate to zero: the idea will be
that if~$\etyped{e}{\kwnz}$, then~$e$ definitely does not evaluate to zero
(that is, if~$e \step^* \kwn{n}$, then~$n \neq 0$).
%
We'll only allow division by expressions that definitely won't evaluate to
zero.

\begin{task}
  Fill in the blanks in the typing rules for {\ezlang}.
  %
  The first two are done for you: any integer can have type~$\kwmz$, but
  it can only have~$\kwnz$ if it's not 0.
  %
  There are two rules for multiplication, depending on whether or not both
  operands are nonzero (think about why).

  \textbf{Hint:} Think about, for each operation (addition, subtraction,
  multiplication, division), whether performing the
  operation on two integers
  can result in zero, depending on whether the two operands can be zero.
  {\em Integers can be negative.}

\end{task}

%% Lucky you, you're using the template: you can just replace each
%% \blank below with the answer.
{
  \centering
  \def \MathparLineskip {\lineskip=0.43cm}
  \begin{mathpar}

    \Rule{T-1}
         {\strut}
         {\etyped{\kwn{n}}{\kwmz}}
    \and
    \Rule{T-2}
         {n \neq 0}
         {\etyped{\kwn{n}}{\kwnz}}
    \and
    \Rule{T-3}
         {\etyped{e_1}{\kwmz}\\
           \etyped{e_2}{\kwmz}}
         {\etyped{e_1 + e_2}{\blank}}
    \and
    \Rule{T-4}
         {\etyped{e_1}{\kwmz}\\
           \etyped{e_2}{\kwmz}}
         {\etyped{e_1 - e_2}{\blank}}
    \and
    \Rule{T-5}
         {\etyped{e_1}{\kwnz}\\
           \etyped{e_2}{\kwnz}}
         {\etyped{e_1 * e_2}{\blank}}
    \and
    \Rule{T-6}
         {\etyped{e_1}{\kwmz}\\
           \etyped{e_2}{\kwmz}}
         {\etyped{e_1 * e_2}{\blank}}
    \and
    \Rule{T-7}
         {\etyped{e_1}{\kwmz}\\
           \etyped{e_2}{\blank}}
         {\etyped{e_1 / e_2}{\blank}}
  \end{mathpar}
}

You may find it concerning that, for example, there's no rule for
$e_1 + e_2$ when~$e_1$ has type~$\kwnz$ and~$e_2$ has type~$\kwmz$.
%
This actually isn't a problem, because\footnote{This is, in fact, a form
  of {\em subtyping}, which you may be familiar with from languages like Java,
  where whenever something has one type, it also has a more general type.
  We'll study this in more detail later in the course.}:

\begin{thm}\label{thm:subtyping}
  For any expression~$e$, if~$\etyped{e}{\kwnz}$ then~$\etyped{e}{\kwmz}$.
\end{thm}


\begin{task}
  Prove Theorem~\ref{thm:subtyping} by induction on the derivation of
  $\etyped{e}{\kwnz}$.

  \textbf{Hint:} If you've set up the typing rules correctly, there should
  only be two cases. If a typing rule has a conclusion of the form
  $\etyped{e}{\kwmz}$, we don't need to consider it because it can't be used
  to derive our assumption~$\etyped{e}{\kwnz}$.
\end{task}

\textbf{Answer:}


The Canonical Forms lemma is similar to before, but now includes the fact that
values of type~$\kwnz$ are not zero.

\begin{lemma}[Canonical Forms]
  If~$\etyped{e}{\tau}$ and~$e \val$, then:
  \begin{enumerate}
  \item If~$\tau = \kwnz$, then~$e = \kwn{n}$ for some~$n \neq 0$.
  \item If~$\tau = \kwmz$, then~$e = \kwn{n}$ for some~$n$.
  \end{enumerate}
\end{lemma}

Here are the statements of Progress and Preservation for {\ezlang}.

\begin{thm}[Progress]
  If~$\etyped{e}{\tau}$ then~$e \val$ or there exists~$e'$ such that
  $e \step e'$.
\end{thm}

\begin{thm}[Preservation]
  If~$\etyped{e}{\tau}$ and~$e \step e'$ then~$\etyped{e'}{\tau}$.
\end{thm}

\begin{task}
  Prove the Progress and Preservation theorems for {\ezlang}.

  \textbf{Hints/Comments -- you'll want to read these.}
  \begin{itemize}
  \item For Progress, you only need to prove the cases for (T-1), (T-2) and
    (T-7)\footnote{This is {\em not} for the same reason that you only need
    to prove two cases for Theorem~\ref{thm:subtyping}; the other cases
    for Progress need to be proven too, this is just me being nice and not
    making you do them :)}.
  \item For Preservation, you only need to prove the cases for (S-3),
    (S-4), (S-5) and (S-6)\footnote{Same.}.
  \item In the (S-5) case for Preservation, if you do inversion on the typing
    rules, there will actually be {\em two} rules you need to consider in your
    inversion and thus, two cases (think about why).
  \item You can use without proof any (true!) facts of basic arithmetic.
  \end{itemize}
\end{task}

\textbf{Answer:}

\begin{task}
  Give an expression in {\ezlang} that {\em does not} perform a division by
  zero, but is still {\em not well-typed} by the above typing rules.
  Suggest a change, addition, refinement, etc., to the type system that could
  make your expression well-typed (but would still rule out expressions that
  divide by zero).
  You don't need to actually implement your suggestion, or even argue that
  it's possible or practical, and it's fine if your solution still rules out
  some ``OK'' programs (programs that don't divide by zero).
  This is just to get you to start thinking about
  how to get what you want with type systems.
\end{task}

\textbf{Answer:}

\end{document}
